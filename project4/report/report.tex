\documentclass{article}

\usepackage{algorithm}
\usepackage{pgfplots}
\usepackage{algpseudocode}
\usepackage[]{algorithm2e}


\begin{document}

\vspace*{3ex}
\begin{flushright}
{\large 12 January 2015}
\end{flushright}

\begin{flushleft}
{\large Jakub Ciecierski\\
}
\end{flushleft}

\hskip3cm

\begin{center}

\Large {\bf An upper triangular dense matrix is split by columns on N processors. Implement the backward substitution. Analyze the speed-up.}

\vskip2ex

{\large Parallel Processing Project 4}

\end{center}

\vskip20ex

\section{Algorithm description}

\begin{algorithm}[H]
  \KwData{Ax=b}
 \For{i=n-1 to 1}{
	\State 	$x[i] = b[i] / A[i,i]$
	
	\For{j=0 to i-1}{
		\State $b[j] = b[j] - x[i]*A[j,i]$
	}\EndFor
 }\EndFor
 
 \caption{Sequential Backwards substitution }
\end{algorithm}

At i-th iteration, the process which has i-th column, computes $x_{i}$
and updates the b vector.
When worker is done computing, he sends the next worker updated vectors b and x.
Since each i-th iteration requires data from the previous iteration, column-wise parallel backward substitution
is scarcely a parallel algorithm and actually works faster sequentially as I will show in the next section.

\section{Analysis of speed up}
I created upper triangular dense matrices using a simple formula:

\begin{algorithm}[H]
 \KwData{n = Dimension of A;
 		val = 1; c = r = 0}
 
 \While{$c > n$}{
	\While{$r > n$}{
			\eIf{$r > c$} {
					\State $A[c*n + r] = 0$
				}{
					\State $A[c*n + r] = val$
					\State $val++$
				}
	 }
 }
 \caption{Column major order Matrix}
\end{algorithm}


with various dimension size and processors count.


\begin{tikzpicture}

	\begin{axis}[
		xlabel= Time of computation in milliseconds,
		ylabel= Processor count,
		height=10cm,
		width=10cm,
		grid=major,
	]

	\addplot coordinates {
			(0.186, 1)
			(1.206, 2)
			(2.211, 4)
			(4.350, 10)
			(9.500, 20)
			(33.500, 50)
	};
	\addlegendentry{n=100}

	\addplot coordinates {
				(18.000, 1)
				(51.000, 2)
				(70.000, 4)
				(90.000, 10)
				(115.500, 20)
				(210.500, 50)
		};
		\addlegendentry{n=1000}
	
	\end{axis}	



\end{tikzpicture}

The plot clearly shows that the algorithm is not really parallel.
In fact the computations are the fastest with one process.
\end{document}
